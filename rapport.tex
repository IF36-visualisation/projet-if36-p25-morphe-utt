% Options for packages loaded elsewhere
\PassOptionsToPackage{unicode}{hyperref}
\PassOptionsToPackage{hyphens}{url}
%
\documentclass[
]{article}
\usepackage{amsmath,amssymb}
\usepackage{iftex}
\ifPDFTeX
  \usepackage[T1]{fontenc}
  \usepackage[utf8]{inputenc}
  \usepackage{textcomp} % provide euro and other symbols
\else % if luatex or xetex
  \usepackage{unicode-math} % this also loads fontspec
  \defaultfontfeatures{Scale=MatchLowercase}
  \defaultfontfeatures[\rmfamily]{Ligatures=TeX,Scale=1}
\fi
\usepackage{lmodern}
\ifPDFTeX\else
  % xetex/luatex font selection
\fi
% Use upquote if available, for straight quotes in verbatim environments
\IfFileExists{upquote.sty}{\usepackage{upquote}}{}
\IfFileExists{microtype.sty}{% use microtype if available
  \usepackage[]{microtype}
  \UseMicrotypeSet[protrusion]{basicmath} % disable protrusion for tt fonts
}{}
\makeatletter
\@ifundefined{KOMAClassName}{% if non-KOMA class
  \IfFileExists{parskip.sty}{%
    \usepackage{parskip}
  }{% else
    \setlength{\parindent}{0pt}
    \setlength{\parskip}{6pt plus 2pt minus 1pt}}
}{% if KOMA class
  \KOMAoptions{parskip=half}}
\makeatother
\usepackage{xcolor}
\usepackage[margin=1in]{geometry}
\usepackage{graphicx}
\makeatletter
\def\maxwidth{\ifdim\Gin@nat@width>\linewidth\linewidth\else\Gin@nat@width\fi}
\def\maxheight{\ifdim\Gin@nat@height>\textheight\textheight\else\Gin@nat@height\fi}
\makeatother
% Scale images if necessary, so that they will not overflow the page
% margins by default, and it is still possible to overwrite the defaults
% using explicit options in \includegraphics[width, height, ...]{}
\setkeys{Gin}{width=\maxwidth,height=\maxheight,keepaspectratio}
% Set default figure placement to htbp
\makeatletter
\def\fps@figure{htbp}
\makeatother
\setlength{\emergencystretch}{3em} % prevent overfull lines
\providecommand{\tightlist}{%
  \setlength{\itemsep}{0pt}\setlength{\parskip}{0pt}}
\setcounter{secnumdepth}{-\maxdimen} % remove section numbering
\ifLuaTeX
  \usepackage{selnolig}  % disable illegal ligatures
\fi
\usepackage{bookmark}
\IfFileExists{xurl.sty}{\usepackage{xurl}}{} % add URL line breaks if available
\urlstyle{same}
\hypersetup{
  hidelinks,
  pdfcreator={LaTeX via pandoc}}

\author{}
\date{\vspace{-2.5em}}

\begin{document}

knitr::opts\_chunk\$set( echo = TRUE, warning = FALSE, message = FALSE,
fig.width = 6, fig.height = 4 ) library(ggplot2) library(dplyr)
library(tidyr)

\subsection{\texorpdfstring{\textbf{LA QUESTION
DONNEE}}{LA QUESTION DONNEE}}\label{la-question-donnee}

\begin{quote}
Existe-t-il une corrélation forte entre certaines propriétés chimiques
et la qualité du vin ?
\end{quote}

\subsection{\texorpdfstring{\textbf{GENERATION DE GRAPHIQUES POUR
REPONDRE A LA QUESTION (vin
rouge)}}{GENERATION DE GRAPHIQUES POUR REPONDRE A LA QUESTION (vin rouge)}}\label{generation-de-graphiques-pour-repondre-a-la-question-vin-rouge}

\begin{quote}
Choix d'un scatter plot pour montrer une corrélation entre un élément
chimique et la note d'un vin
\end{quote}

\subsubsection{Chargement des données}\label{chargement-des-donnuxe9es}

redwine \textless-
read.csv(``\textasciitilde/projet-if36-p25-morphe-utt/data/red\_wine.csv'',
header = TRUE, sep = `;') whitewine \textless-
read.csv(``\textasciitilde/projet-if36-p25-morphe-utt/data/white\_wine.csv'',
header = TRUE, sep = `;')

\subsubsection{Etape 1 : On ne garde que les variables qui sont reliées
à de la
chimie}\label{etape-1-on-ne-garde-que-les-variables-qui-sont-reliuxe9es-uxe0-de-la-chimie}

variables\_chimiques \textless- c(``fixed\_acidity'',
``volatile\_acidity'', ``citric\_acid'', ``residual\_sugar'',
``chlorides'', ``free\_sulfur\_dioxide'', ``total\_sulfur\_dioxide'',
``density'', ``pH'', ``sulphates'', ``alcohol'')

\subsubsection{Etape 1.5 : Ajout des unités aux catégories
gardées}\label{etape-1.5-ajout-des-unituxe9s-aux-catuxe9gories-garduxe9es}

unit\_labels \textless- c( ``fixed\_acidity'' = ``Acidité fixe
(g/dm³)'', ``volatile\_acidity'' = ``Acidité volatile (g/dm³)'',
``citric\_acid'' = ``Acide citrique (g/dm³)'', ``residual\_sugar'' =
``Sucre résiduel (g/dm³)'', ``chlorides'' = ``Chlorures (g/dm³)'',
``free\_sulfur\_dioxide'' = ``SO2 libre (mg/dm³)'',
``total\_sulfur\_dioxide'' = ``SO2 total (mg/dm³)'', ``density'' =
``Densité (g/cm³)'', ``pH'' = ``pH'', ``sulphates'' = ``Sulfates
(g/dm³)'', ``alcohol'' = ``Alcool (\% vol.)'', )

\subsubsection{Etape 2 : On factorise nos différentes variables et leurs
valeurs dans un tableau à 2
dimensions.}\label{etape-2-on-factorise-nos-diffuxe9rentes-variables-et-leurs-valeurs-dans-un-tableau-uxe0-2-dimensions.}

\paragraph{ça va permettre de générer pleins de graphiques en une seule
fois plutot que d'en générer plusieurs à la
main}\label{uxe7a-va-permettre-de-guxe9nuxe9rer-pleins-de-graphiques-en-une-seule-fois-plutot-que-den-guxe9nuxe9rer-plusieurs-uxe0-la-main}

redwine2 \textless- redwine \%\textgreater\% select(quality,
all\_of(variables\_chimiques)) \%\textgreater\% pivot\_longer( cols =
all\_of(variables\_chimiques), names\_to = ``variable'', values\_to =
``valeur'' )

\subsubsection{Etape 3 : Génération des scatters plots avec le tableau
factorisé
B)}\label{etape-3-guxe9nuxe9ration-des-scatters-plots-avec-le-tableau-factorisuxe9-b}

ggplot(redwine2, aes(x = valeur, y = quality)) + \# Génération des
points geom\_point(alpha = 0.3, shape = 1) + \# Génération de la droite
de regression avec son intervalle de confiance geom\_smooth(method =
``lm'', se = TRUE, color = ``darkred'') + \# Sert pour la création de
nos graphiques, c'est ce qui permet la répartion sur 3 colonnes \# mais
aussi le fait que les abcisses soient libre
facet\_wrap(\textasciitilde{} variable, scales = ``free\_x'', ncol = 3,
labeller = labeller(variable = unit\_labels)) + \# Utilise un thème
épuré, pour faire ressortir la droite de regression je trouve ça cool
theme\_minimal(base\_size = 12) + \# Gestion des labels labs( title =
``Relations entre une variable chimique d'un vin rouge et sa note de
qualité'', x = ``Valeur de la variable chimique'', y = ``Qualité (note
sur 10 attribuée par un ou des expert(s))'' )

\subsection{\texorpdfstring{\textbf{GENERATION DE GRAPHIQUES POUR
REPONDRE A LA QUESTION (vin
blanc)}}{GENERATION DE GRAPHIQUES POUR REPONDRE A LA QUESTION (vin blanc)}}\label{generation-de-graphiques-pour-repondre-a-la-question-vin-blanc}

\subsubsection{Etape 2}\label{etape-2}

whitewine2 \textless- whitewine \%\textgreater\% select(quality,
all\_of(variables\_chimiques)) \%\textgreater\% pivot\_longer( cols =
all\_of(variables\_chimiques), names\_to = ``variable'', values\_to =
``valeur'' )

\subsubsection{Etape 3 :}\label{etape-3}

ggplot(whitewine2, aes(x = valeur, y = quality)) + \# Génération des
points geom\_point(alpha = 0.3, shape = 1) + \# Génération de la droite
de regression avec son intervalle de confiance geom\_smooth(method =
``lm'', se = TRUE, color = ``darkred'') + \# Sert pour la création de
nos graphiques, c'est ce qui permet la répartion sur 3 colonnes \# mais
aussi le fait que les abcisses soient libre
facet\_wrap(\textasciitilde{} variable, scales = ``free\_x'', ncol = 3,
labeller = labeller(variable = unit\_labels)) + \# Utilise un thème
épuré, pour faire ressortir la droite de regression je trouve ça cool
theme\_minimal(base\_size = 12) + \# Gestion des labels labs( title =
``Relations entre une variable chimique d'un vin blanc et sa note de
qualité'', x = ``Valeur de la variable chimique'', y = ``Qualité (note
sur 10 attribuée par un ou des expert(s))'' )

\subsection{\texorpdfstring{\textbf{ANALYSE ET REPONSE A LA QUESTION
DONNEE}}{ANALYSE ET REPONSE A LA QUESTION DONNEE}}\label{analyse-et-reponse-a-la-question-donnee}

\textbf{VIN ROUGE :} La droite de regression est stable quasiment de
partout, on peut donc en conclure qu'il n'y a pas de lien fort entre les
éléments chimiques et la note du vin. Une des hypothèses qu'on pourrait
tirer serait que la note pourrait être attribuée en fonction du
ressentit des experts. Par exemple lors d'une dégustation, une note
peut-être attribuée en fonction du gout, de l'odorat, voire aussi du
visuel si il y a des dépots en surface etc\ldots{} Certainement que la
provenance du vin joue aussi.

\textbf{VIN BLANC :} Concernant le vin blanc la même analyse peut-être
appliquée, la droite de regression est stable quasiment partout.
Cependant un lien plus fort peut-être observé entre le pH et la note,
ainsi que le sucre residuel et les sulfates. En effet la droite de
regression tend de manière négative, donc plus elle descend dans les
notes et plus ces valeurs chimiques sont élevées dans le vin. Donc si on
reprend notre analyse concernant le vin rouge, on pourrait en déduire
que ces variables jouent plus sur les variables de ressentit des experts
(gout, visuel etc\ldots) que le reste. Plus elles sont élevées dans un
vin, plus elles impactes sur la qualité du vin, et plus ce dernier a une
note médiocre

\subsection{\texorpdfstring{\textbf{CONCLUSION}}{CONCLUSION}}\label{conclusion}

\textbf{En conclusion : } Il existe bel est bien une corrélation entre
certains éléments chimiques et la note du vin, puisque l'évolution des
éléments chimiques impactent la note et inversement. En revanche il ne
peut pas être catégorisé comme lien fort, puisque aucune variable
chimique à elle seule impact de manière majoritaire la note d'un vin en
particulier. AVec cette analyse, on comprend surtout que les variables
chimiques d'un vin forment un tout, et que c'est leurs associations qui
font varier un vin et sa qualité !

\end{document}
